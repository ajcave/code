
\documentclass[11pt]{article}
\textheight=9.5in
\textwidth=7in
\oddsidemargin=-.34in
\topmargin=-.75in
\renewcommand{\baselinestretch}{1.2}
\pagestyle{plain}

\begin{document}
\begin{center}
\large {\bf NSERC PHD STUDENT RESEARCH PROGRESS REPORT}
\end{center}

\medskip


\textbf{Name:} Andrew Cave

\textbf{Dates of Applicable Time Period}:

\textbf{From:} May 2013            \ \ \textbf{To:} March 2014


\begin{enumerate}

\item \textbf{Publications:}
  \begin{enumerate}
  \item \textbf{Journals:} None

  \item \textbf{Conferences:}
Andrew Cave, Francisco Ferreira, Prakash Panangaden and Brigitte Pientka. Fair reactive programming to 41st ACM SIGPLAN-SIGACT Symposium on Principles of Programming Languages (POPL'14) (refereed).
  \item \textbf{Other:}  
   Andrew Cave and Brigitte Pientka. First-class substitutions in contextual type theory. Proceedings of the Eighth ACM SIGPLAN international workshop on Logical frameworks \& meta-languages: theory \& practice (LFMTP'13) (refereed)

  \end{enumerate}

\item \textbf{Talks Given at McGill or Elsewhere, Workshops Attended, Research Visits:} 
Attended Workshop on Formal Meta-Theory at INRIA March 2013, presented Fair Reactive Programming.

Presented First-class substitutions in contextual type theory at LFMTP'13 in September 2013.

Presented Fair Reactive Programming at POPL'14 in January 2014.
\item \textbf{Summary of Research Progress}
January 2014, submitted to ITP 2014: Andrew Cave and Brigitte Pientka. Mechanizing Logical Relations using Contextual Type Theory.

The LFMTP'13 paper and ITP'14 submission represent progress on extending contextual type theory to support effective mechanization of logical relations proofs for simply typed languages.

\end{enumerate}

\end{document}

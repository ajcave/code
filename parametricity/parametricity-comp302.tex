\documentclass{article}
\usepackage{listings}
\title{Parametricity and Modular reasoning}
\begin{document}
\maketitle
\begin{lstlisting}
module type SwitchType = sig
 type t
 val init : t
 val flip : t -> t
 val isOn : t -> bool
end

module Switch1 : SwitchType = struct
 type t = int
 let init = 1 (* positive means `off', negative means `on' *)
 let flip n = -2*n
 let isOn n = (n < 0)
end

module Switch2 : SwitchType = struct
 type t = bool
 let init = false (* false means `off', true means `on' *)
 let flip b = not b
 let isOn b = b
end
\end{lstlisting}

Intuitively, it seems that we can't tell these two implementations are
equivalent. Because the type t is abstract, and all we can do is flip
the initial switch a few times, and then read its result. It shouldn't
matter whether whether we use \lstinline{Switch1} or
\lstinline{Switch2} (ignoring overflow).

- Enforced by type system

- Parametricity doesn't quite hold in Ocaml. Equality and Obj.magic break it.
- Design decision, simplifies ...?. But as a module implementor, it's difficult/impossible to be sure that a change you make won't break someone else's code
- Java breaks parametricity...
 - Doesn't have modules
 - Does have generics
 

\end{document}
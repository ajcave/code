\documentclass{article}
\usepackage{alltt}
\usepackage{amsmath, amsthm, amssymb}

\title{Contexts in Abella}
\author{Andrew Cave}

\begin{document}
\maketitle
Explain in context of church rosser?

Problems:
\begin{enumerate}
\item {\tt member (of (app M N) T) G} should be discarded
\item Freshness: {\tt member (of x (T x)) (G x)} - we know that  {\tt x}
  does not occur in {\tt T}. Example? How do we handle this? 
\item Pruning: {\tt member (E x) G} entails {\tt E}'s dependency on
  {\tt x} is vacuous. How to handle?
\item Know one ({\tt pr x x}) get others ({\tt cd x x} and {\tt notlam x}).
\item Functional relationships (only one type per variable). We
  certainly know that each variable can map to exactly one block by
  freshness.
\item Strengthening. We would like to consider {\tt wf} in only
  contexts containing well-formedness. We would like to associate to
  each family a world.
\end{enumerate}

Block declarations:

n-ary worlds and blocks are defined: 
\begin{align*}
W_n &:= B_n | ... | B_n \\
B_n &:= \exists \vec{T}, \nabla \vec{x}, \underbrace{tl(\vec{x}, \vec{T}); ...;
tl(\vec{x}, \vec{T})}_{n} \\
tl(\vec{x}, \vec{T}) &:= t(\vec{x}, \vec{T}), ..., t(\vec{x}, \vec{T}) \\
t(\vec{x}, \vec{T}) &:= \text{term with variables drawn from } \vec{x} \text{
  and } \vec{T}
\end{align*}

- Nabla may not be the right binder
- Note that the $T$s are outside the scope of the $\nabla$ and hence
any dependency must be explicitly mentioned.
- Why? Because this jives better with translation into context
relations? 

Elaboration into relations:

Block b := nabla x, (pr x x; cd x x, notlam x)
World w := b1 | b2 | ... | bn

pr x x :- var x
wf x :- var x

Comparisong of church rosser with Agda

\section{Strengthening}
We would like to consider {\tt wf} in only
contexts containing well-formedness assumptions.

We would like to associate to each ``family'' a world.
Do I have a better (simpler) example than PTS?
When we do case analysis on {\tt {G |- of M T}} one of the cases yields
{\tt {G |- of M U}} and {\tt {G |- eq T U}}. But we would like {\tt eq}
  to be immediately stated in its world: {\tt {W |- eq T U}} where
  {\tt ctxRel G W}. So does case analysis yield a new world? If it
  yields a new {\tt W} every time, that's bad. So case analysis should
  work with respect to a ``world'' such as {\tt ctxRel G W}.

Conversely, when we have {\tt ctxRel G W}, {\tt {G |- of M U}}, {\tt
  {W |- eq U T}}, we want to be able to immediately prove (via search)
{\tt {G |- of M T}}. Key appears to be {\tt ctxRel} is
\emph{functional} from G to W. Not just existence, but uniqueness.

Motivate this: We want to state theorems of the form {\tt thm : wfctx W ->
  {W |- eq T U} -> ...}. So when we case analyze a {\tt {G |- of M T}}
and obtain a {\tt {G |- eq T U}}, that's bad. We have to do work to
connect it to {\tt thm}.

Explain how I imagine this working better in, say, Coq.

One would like to enforce the property that {\tt {G |- f t1 ... tn}} is
only ``well-typed'' if {\tt G} is a world of the type associated to
the family {\tt f}. 

(Look at how much work eq\_typing.thm does just for this)

In particular, this generates the need for theorems of this form:

\begin{verbatim}
Theorem typing_wf_ctx_is_wf_ctx : forall G D, typing_wf_ctx G D -> wf_ctx D.
\end{verbatim}

Concrete example: 
\begin{verbatim}
Theorem eq_for_inv : forall W U V T1 T2,
   wf_ctx W
-> {W |- wf (for U T1)}
-> {W |- wf (for V T2)}
-> {W |- eq (for U T1) (for V T2)}
-> {W |- eq U V} /\ nabla x, {W, wf x |- eq (T1 x) (T2 x)}.
\end{verbatim}

In order to apply this inside:

\begin{verbatim}
Theorem subj_red : forall G M M' T,
   ctx G
-> {G |- of M T}
-> {G |- step M M'}
-> {G |- of M' T}.
\end{verbatim}

in which we will have assumptions
{\tt {G |- eq (for U T1) (for V T1)}}, we have to strengthen.

We could state everything in ``large enough'' contexts (e.g. we could
state {\tt eq\_for\_inv}) in the contexts G with {\tt ctx G}, but this is
bad for modularity.

Comment that we can handle ``side conditions'' like ``of U K'' in PTS
by adding (forall X U, member (of X U) G -> {G |- of U K})

\end{document}
\documentclass[usenames,dvipsnames]{beamer}
\setbeamertemplate{footline}[frame number]
\usepackage{alltt}

\usepackage{tikz}
\usetikzlibrary{arrows,automata}
\setbeamertemplate{navigation symbols}{}%remove navigation symbols
\usepackage{pifont}% http://ctan.org/pkg/pifont
\newcommand{\cmark}{\ding{51}}%
\newcommand{\xmark}{\ding{55}}%
\usepackage{ textcomp }
\usepackage{array}

\usepackage{srcltx}
\usepackage[tikz]{bclogo}


\providecommand{\term}[1]{\mbox{\small{\textsf{#1}}}}
\usepackage{listings}

\newcommand{\const}[1]{\textbf{#1}}
\newcommand{\bnfas}{\mathrel{::=}}
\newcommand{\bnfalt}{\mathrel{\mid}}
\newcommand{\arrow}{\to}
\newcommand{\lam}[1]{\lambda #1. }
\usepackage{proof}
\newcommand{\entails}{\vdash}
\newcommand{\der}{\vdash}
\newcommand{\stepsto}{\longrightarrow}
\newcommand{\id}{\text{id}}
\setbeamertemplate{itemize item}{\scriptsize\raise1.25pt\hbox{\donotcoloroutermaths$\blacktriangleright$}}
\setbeamertemplate{itemize subitem}{\tiny\raise1.5pt\hbox{\donotcoloroutermaths$\bullet$}}
\setbeamertemplate{itemize subsubitem}{\tiny\raise1.5pt\hbox{\donotcoloroutermaths$-$}}
\definecolor{light-gray}{gray}{0.5}

\lstloadlanguages{Ltl}
\lstset{language=Ltl}  % Added to local lstlang2.sty

% Set up listings "literate" keyword stuff (for \lstset below)
\newdimen\zzlistingsize
\newdimen\zzlistingsizedefault
\zzlistingsizedefault=9pt
\zzlistingsize=\zzlistingsizedefault
\global\def\InsideComment{0}
\newcommand{\Lstbasicstyle}{\fontsize{\zzlistingsize}{1.05\zzlistingsize}\ttfamily}
\newcommand{\keywordFmt}{\fontsize{0.9\zzlistingsize}{1.0\zzlistingsize}\bf}
\newcommand{\smartkeywordFmt}{\if0\InsideComment\keywordFmt\fi}
\newcommand{\commentFmt}{\def\InsideComment{1}\fontsize{0.95\zzlistingsize}{1.0\zzlistingsize}\rmfamily\slshape}

\newcommand{\LST}{\setlistingsize{\zzlistingsizedefault}}

\newlength{\zzlstwidth}
\newcommand{\setlistingsize}[1]{\zzlistingsize=#1%
\settowidth{\zzlstwidth}{{\Lstbasicstyle~}}}
\setlistingsize{\zzlistingsizedefault}
\renewcommand{\bar}{\overline}

% The order of the "literate" definitions is significant:
%   later definitions shadow earlier ones.  The \\Pi definition must come
%   *after* the \\ definition, or the first part of \\Pi --- that is, \\ --- will
%   be matched, and instead of $\Pi$ you'll get $\lambda Pi$.
%
\lstset{literate={->}{{$\rightarrow~$}}1 %
                 {=>}{{$\Rightarrow~$}}2 %
                 {id}{{{\smartkeywordFmt id}}}1 % 3 $~$
                 {\\}{{$\lambda$}}1 %
                 {\\bullet}{{$\bullet$}}1 %
                 {\\always}{{$\always$}}1 %
                 {\\next}{{$\next$}}1 %
                 {\\andop}{{$\andop$}}1 %
                 {\\nexti}{{$\nexti$}}1 %
                 {\\eventually}{{$\eventually$}}1 %
                 {\\until}{{$\until$}}2 %
                 {\\onceuntil}{{$\onceuntil$}}2 %
                 {\\orop}{{$\orop$}}1 %
                 {\\times}{{$\times$}}1 %
%                 {mlam}{{$\lambda^{\scriptscriptstyle\Box}$}}1 %
                 {mlam}{{$\lambda$}}1 %
                 {FN}{{$\Lambda$}}1 %
                 {<<}{\color{red}}1 %
                 {>>}{\color{black}}1 %
               ,
               columns=[l]flexible,keepspaces,
               basewidth=\zzlstwidth,
               basicstyle=\Lstbasicstyle,
               keywordstyle=\keywordFmt,
               identifierstyle=\relax,
%               stringstyle=\relax,
               commentstyle=\commentFmt,
               breaklines=true,
               breakatwhitespace=true,   % doesn't do anything (?!)
               mathescape=true,   % interprets $...$ in listing as math mode
%               tabsize=8,
               texcl=false}


\author{Andrew Cave}
\title[{\makebox[.45\paperwidth]{Dependent Type Theory for Contextual Reasoning\hfill%
       \insertframenumber/\inserttotalframenumber}}]{Dependent Type Theory for Contextual Reasoning} 
\begin{document}

\begin{frame}
\titlepage
\end{frame}

\begin{frame}{Introduction}
\begin{itemize}
\item Proofs can be long and error-prone to check by hand
\pause \item e.g. Feit Thompson theorem: proof spans 2 books, 350 pages
\pause \item Lots of work on languages for machine-checkable proofs (Automath, Mizar, NuPRL, Coq, ...)
\pause \item Is writing machine-checkable proofs a good idea?
\pause \item {\color{green} Pro:} can be easier to trust
\pause \item {\color{green} Pro:} force rigorous thinking
\pause \item {\color{green} Pro:} coming up with proofs can be \emph{interactive}
% \begin{itemize}
% \item System: ``Goal is X''
% \item Human prover: ``Use lemma foo''
% \item System: ``New subgoals: Y and Z''
% \end{itemize}
\pause \item {\color{red} Con:} can require substantial ``boilerplate''
\begin{itemize}
\item Makes proofs harder to write \emph{and} read
\item How can this be improved?
% \item Need to somehow equip proof checker with domain-specific procedures
\end{itemize}
\end{itemize}
\end{frame}

\begin{frame}{Formalizing programming language theory}
\begin{itemize}
\item Seems promising!
\item Lots of interest already (e.g. POPLMark challenge)
% \begin{itemize}
% \item Inductive reasoning is well understood
% \item Lots of cases! Can be
% tedious and error-prone to write \& verify by hand.
% \end{itemize}
\item More interest than in math
\pause
\item Overhead: formalizing variable binding and substitution
\end{itemize}
\pause
\begin{quote}
When doing the formalization, I discovered that the core part of the
proof [...] is fairly straightforward and
only requires a good understanding of the paper version. However, in
completing the proof I observed that in certain places I had to invest
much more work than expected, e.g. proving lemmas about substitution
and weakening. -- Thorsten Altenkirch, 1993

   (formalizing normalization of System F)
\end{quote}
% \pause
% \begin{itemize}
% \item 20 years later: This proof still hasn't been done in a less painful way(?!)
% \end{itemize}
\end{frame}

\begin{frame}{Thesis}
\begin{quote}Building a logical framework into dependent type
theory is possible and makes mechanization of state-of-the-art programming
language metatheory more effective.
\end{quote} % Say: i.e. we want to be able to formalize our results without substantially more effort than we would have to invest on paper

\pause\begin{quote}{\normalfont More generally: Learn something about how
to make mechanization of proofs outside of programming languages and
logic more practical?}\end{quote}
\end{frame}

\begin{frame}[fragile]{Representing languages: The basics}
% Focus here on tal
\begin{tabular}{p{2.8cm}l}
\fbox{Syntax}\vfill &
\begin{array}[t]{lrcl}
\mbox{Types} & A,B & \bnfas & {\const b} \bnfalt A \arrow B \\
\mbox{Terms} & M,N & \bnfas & {\const c} \bnfalt {\color{purple}x} \bnfalt \lambda {\color{purple} x.} M \bnfalt M\;N \\
\mbox{Values} & V & \bnfas & {\const c} \bnfalt \lam x M
\end{array}
\\[3em]
\pause
\vspace{-2.3em}\fbox{Typing $\Gamma \vdash M : A$} &
\begin{array}[t]{l}
\infer{\Gamma \vdash x : A}{\Gamma(x) = A} \quad
\infer{\Gamma \entails \lam x M : A \arrow B}{\color{purple}\Gamma,x{:}A \der M : B}
\\[0.4em]
\infer{\Gamma \entails {\const c} : {\const b}}{}
\quad
\infer{\Gamma \entails M\; N : B}{\Gamma \entails M : A \arrow B \quad \Gamma \entails N : A}
\end{array}
\\[3em]
\pause
\vspace{-2.3em}\fbox{Op. Sem.} & 
$\infer{M\;N \stepsto M'\;N}{M \stepsto M'} \quad \infer{(\lam x M) N \stepsto {\color{purple}[N/x]M}}{} $ \\[2em]
\end{tabular}

\hfill $M$ halts if $\exists V.\, M \stepsto^* V$ \hfill
\end{frame}

\begin{frame}{Background: Higher-order abstract syntax}
\begin{itemize}
\item Lots of approaches to support binding and substitution
\item Higher-order abstract syntax (HOAS) very effective
\begin{itemize}
\item Idea: represent binding and substitution using functions and application from a meta-language
\end{itemize}
\item The logical framework LF employs HOAS
\item Twelf [Pfenning and Sch\"urmann '99] implements LF
\begin{itemize}
\item Very effective for basic results in programming language theory (type soundness, confluence)
% \item Less clear how to formulate induction
% over HOAS representations
\item Steep price: very weak logic. Can only express $\forall\exists$ statements
% \item Abella: first order logic (weak)
% \item Hybrid, nominal logic, locally nameless, etc: less support for substitution
\item Cannot express advanced results (normalization, parametricity)
\end{itemize}
\end{itemize}
\end{frame}

\begin{frame}[fragile]{Example: HOAS encoding}
\begin{tabular}{p{3.2cm}l}
\vspace{-2.7em}\fbox{Syntax} &
\begin{lstlisting}
b : tp.
_=>_ : tp -> tp -> tp.

c : tm.
lam : ${\color{purple}(\text{tm} \rightarrow \text{tm})}$ -> tm.
app : tm -> tm -> tm.
\end{lstlisting}
\\[3em]
\pause
\vspace{-2.7em}\fbox{Typing \lstinline{oft M T}} &
\begin{lstlisting}
oft/b : oft c b.
oft/lam : ${\color{purple}\text{(\{x:tm\} oft x A -> oft (M x) B)}}$
          -> oft (lam M) (A => B).
oft/app : oft M (A => B) -> oft N A
          -> oft (app M N) B.
\end{lstlisting}
\\[3em]
\pause
\vspace{-2em}\fbox{Op. Sem. \lstinline{step M N}} & 
\begin{lstlisting}
step/app : step M M'
        -> step (app M N) (app M' N).
step/beta : step (app (lam M) N) $\color{purple}\text{(M N)}$.
\end{lstlisting}
\end{tabular}
\end{frame}

\begin{frame}{Background: Two-level approaches}
\begin{itemize}
\item HOAS representation on one level (e.g. LF)
\item Richer logic/type theory at the next
\item e.g. implemented in Abella [Gacek'08], Beluga [Pientka'08], Delphin [Poswolsky and Sch\"urmann '08]
\item $\approx$ indexed type theory/first-order logic with induction
\item Still not enough to directly express many advanced results...
\pause \item Question: can we have more powerful logics?
\end{itemize}
\end{frame}

\begin{frame}[fragile]{Example: Weak normalization}
\begin{itemize}
\item Theorem: If $\vdash M : A$ then $M$ halts
\pause
\begin{enumerate}
\item Define predicate $\mathcal{R}_A$
% \item Show if $M' \in \mathcal{R}_A$ and $M \stepsto M'$ then $M \in \mathcal{R}_A$
\item If $M \in \mathcal{R}_A$ then $M$ halts
\item If $\vdash M : A$ then $M \in \mathcal{R}_A$
\pause

Generalize: {\color{purple}If $\Gamma \vdash M : A$ and $\sigma \in \mathcal{R}_\Gamma$ then $[\sigma]M \in \mathcal{R}_A$}

$\,$

(where $\Gamma$ of the form $x_1{:}B_1,...,x_n{:}B_n$ and $\sigma$ of the form $N_1/x_1,...,N_n/x_n$ and $\sigma \in \mathcal{R}_\Gamma$ means $N_1 \in \mathcal{R}_{B_1},...,N_n \in \mathcal{R}_{B_n}$)
\end{enumerate}
\pause \item Representative of wide variety of advanced proofs

(logical relations)
\end{itemize}
% We prove the more general result that every term of type $A$ is \textbf{reducible} at type $A$, where reducibility is defined:
% \[
% \begin{array}{l}
% \mathcal{R}_{\const i} = \{M\; |\; M \text{ halts}\} \\
% \mathcal{R}_{A \arrow B} = \{M\; |\; M \text{ halts and } {\color{purple} \forall N
% \in \mathcal{R}_A}, (M\; N) \in \mathcal{R}_B \}
% \end{array}
% \]

% \pause
% \begin{lemma}[Backward closure]
% If $M' \in \mathcal{R}_A$ and $M \stepsto M'$ then $M \in \mathcal{R}_A$
% \end{lemma}
\end{frame}

\begin{frame}{Definition of $\mathcal{R}$}
\[
\begin{array}{l}
\mathcal{R}_{\const b} = \{M\; |\; M \text{ halts}\} \\
\mathcal{R}_{A \arrow B} = \{M\; |\; M \text{ halts and } {\color{purple}\forall N
\in \mathcal{R}_A}, (M\; N) \in \mathcal{R}_B \}
\end{array}
\]
\end{frame}

\begin{frame}{Main theorem}

If $\Gamma \vdash M : A$ and $\sigma \in \mathcal{R}_\Gamma$ then
$[\sigma]M \in \mathcal{R}_A$.
\pause
\begin{proof}
By induction. 
Case $\begin{array}{l}\infer{\Gamma \der \lam x M : A \arrow
    B}{\Gamma,x{:}A \der M : B}\end{array}$: Given $N \in \mathcal{R}_A$.

\begin{enumerate}
\pause \item $(\sigma,N/x) \in \mathcal{R}_{\Gamma,x:A}$ \hfill (by definition)
\pause \item $[\sigma,N/x]M \in \mathcal{R}_B$ \hfill (by I.H.)
\pause \item {\color{purple} $[\sigma,N/x]M = [N/x][\sigma,x/x]M$ \hfill (property of substitution)}
\pause \item $[N/x][\sigma,x/x]M \in \mathcal{R}_B$ \hfill (previous lines)
\pause \item $(\lam x ([\sigma,x/x]M))\; N \longrightarrow [N/x][\sigma,x/x]M$ \hfill ($\beta$ reduction)
\pause \item $(\lam x ([\sigma,x/x]M))\; N \in \mathcal{R}_B$ \hfill (backward closure)
\pause \item $(\lam x ([\sigma,x/x]M)) \in \mathcal{R}_{A \arrow B}$ \hfill (by definition)
\pause \item {\color{purple} $[\sigma](\lam x M) \in \mathcal{R}_{A \arrow B}$ \hfill (property of substitution)}
\end{enumerate}
\end{proof}
Proving these properties of substitution requires $\approx$ dozen lemmas!
\end{frame}

% \begin{frame}{Motivating Example: Proof by logical relations}{Problems to solve in a formalization}
% \begin{itemize}
% \item Representing binding: $\lambda x. M$?
% \item Definition of $\mathcal{R}$?
% \item Definition of substitution $[\sigma]M$?
% \item Quantification over contexts $\Gamma$?
% \item Proving equations: $[N/x][\sigma,x/x]M = [\sigma,N/x]M$?
% \end{itemize}
% \end{frame}

\newcommand{\good}{{\color{green}\textbf{\checkmark}}}
\newcommand{\bad}{{\color{red}\textbf{\xmark}}}
\newcommand{\ok}{{\color{orange}$\mathbf{\sim}$}}

\begin{frame}[fragile]{Support from proof environments}
\begin{tabular}{| p{4.4cm} | l | l | l |}
\hline
 & Type theory & Twelf & Two-level \\
 & (Coq) & & (Beluga) \\ \hline
Def. Binding: $\lambda x. M$ & \ok & \good & \good \\ \hline
Ind. predicates e.g. $\mathcal{R}_{\Gamma}$ & \good & \bad & \only<1>{?}\only<2>{\good$^1$} \\ \hline
Def. of $\mathcal{R}_A$ & \good & \bad & \only<1>{?}\only<2>{\ok$^1$} \\ \hline
Def. subst. $[N/x]M$ & \ok & \good & \good \\ \hline
Def. subst. $[\sigma]M$ & \ok & \ok & \only<1>{?}\only<2>{\good$^2$} \\ \hline
Directly inspect contexts & \good & \bad & \good \\ \hline
Equations e.g. \begin{small}$[N/x][\sigma,x/x]M = [\sigma,N/x]M$\end{small} & \bad & \ok & \only<1>{?}\only<2>{\good$^2$} \\ \hline
Reasoning about functions & \good & \bad & ? \\ \hline
\end{tabular}

$\,$

\only<2>{$^1$ Indexed recursive types [Cave and Pientka POPL'12]

$^2$ First-class substitutions [Cave and Pientka LFMTP'13]}
\end{frame}

% \begin{frame}{Motivating Example: Proof by logical relations}{Properties of substitution}
% \begin{itemize}
% \item First, names are tricky.
% \item de Bruijn indices tend to work better for brute-force formalization than names
% \begin{lemma}
% $[N/x][\sigma,x/x]M = [\sigma,N/x]M$ (provided $x$ fresh for $\sigma$)
% \end{lemma}
% becomes
% \begin{lemma}
% $[ id , N ] [ \uparrow \sigma, 1 ] M = [ \sigma , N ] M$
% \end{lemma}
% \end{itemize}
% % \begin{lemma}
% %  [ [ \(\sigma1\) ]' \(\sigma2\) ] = [ \(\sigma1\) ] \(\circ\) [ \(\sigma2\) ]
% % \end{lemma}
% % lem2 : [ \(\sigma\) , M ] \(\circ\) \(\uparrow\) = [ \(\sigma\) ]
% % lem3 : [ id ]' \(\sigma\) = \(\sigma\)
% % lem4 : [ id , N ] \(\circ\) [ \(\uparrow\) \(\sigma\), 1 ] = [ \(\sigma\) , N ]
% % ...
% % \end{alltt}
% This is tedious to do for each language we study!
% \end{frame}


% \begin{frame}{My progress to date}
% \begin{itemize}
% \item Indexed recursive types [Cave and Pientka '12]:  
% \begin{itemize}
% \item Part way to supporting definition of $\mathcal{R}$
% \end{itemize}
% \item First-class substitutions [Cave and Pientka '13]:
% \begin{itemize}
% \item 
% \end{itemize}
% \end{itemize}
% \end{frame}

\begin{frame}{Indexed recursive types [Cave and Pientka POPL'12]}
\begin{itemize}
\item Can express forms of predicates that LF can't e.g.
\begin{itemize}
\item certain subexpressions are \emph{closed}
\item relations between contexts
\end{itemize}
\item Richer forms of induction
\item Result: type soundness and decidability of typechecking
\item Implemented in the Beluga system
\item Good for \emph{programming}
\item Not so good for \emph{reasoning}: naive treatment is inconsistent!
\item \textbf{But} enables \textbf{experimental} definitions of logical relations
\item Not ideal: relations between contexts are typically \emph{functional} relations: why not treat them as \emph{functions}
\end{itemize}
\end{frame}

\begin{frame}[fragile]{First class substitutions [Cave and Pientka LFMTP'13]}
\begin{itemize}
\item Adds distinguished type of substitutions $\sigma$
\item Equational theory \textbf{built-in} to type checking
\item User does not need to prove any equations!
\item Result: Decidability of typechecking w/ equational theory
\item Example: remarkably succinct proof of weak normalization!
\begin{lstlisting}
rec thm : {M:[$\Gamma$ $\vdash$ tm A]} RedS [$\vdash$ $\sigma$] -> Red [$\vdash$ A] [$\vdash$ M $\sigma$] =
mlam M => fn rs => case M of
| [$\Gamma$ $\vdash$ lam (\x. M1)] =>
  Arr [$\vdash$ h/val s/refl val/lam]
   (mlam N => fn rN => closed [$\vdash$ s/beta]
    (thm [$\Gamma$,x $\vdash$ M1] (Cons rs rN)))
...
\end{lstlisting}
\item Uses ``unsound in general'' recursive types to define $\mathcal{R}_A$
\item Future work: fix this
% \item Convenient for logical relations: need to quantify over ``closing substitutions''
% \begin{theorem}[Fundamental theorem]
% If $\Gamma \vdash M : A$ {\color{purple} and $\sigma \in \mathcal{R}_\Gamma$} then
% $[\sigma]M \in \mathcal{R}_A$.
% \end{theorem}
% \item New equations to solve: $[N/x][\sigma,x/x]M = [\sigma,N/x]M$ provided $x \not\in fv(\sigma)$
\end{itemize}
\end{frame}

\begin{frame}{Draft: Logical relations in contextual type theory}
\begin{itemize}
\item Formalizes more involved logical relations proof
\begin{itemize}
\item completeness of an algorithmic equality for STLC
\end{itemize}
\item Core of more advanced results (decidability of typechecking)
\item Formalization is ``easy''!
\item Substantially smaller than other formalizations
\item Not easily done in other two-level systems
% \item Lacks check that definitions are reasonable (positivity)
\item Demonstrates that previous work scales
\end{itemize}
\end{frame}

\begin{frame}{Open problems}
\begin{enumerate}
\item Properly justify definition of logical relations
\item Direct reasoning about functions
\item Scale to even more interesting proofs
\end{enumerate}
\end{frame}

\begin{frame}{Proposal: Move from indexed types to dependent types}
\begin{itemize}
\item \emph{Some} richer theory is necessary
\item Dependent type theory [Martin-L\"of '73] unifies programming with proving!
\item On par with set theory and higher-order logic
\item Includes \emph{definitional equalities}: checked automatically!
\begin{itemize}
\item Extremely convenient!
\item Computation ($\beta$) rules
\item Conceivably other rules?
\end{itemize}
\item Direct reasoning about functions
\begin{itemize}
\item Major practical impact!
\item e.g. concatenation of contexts
\end{itemize}
\item \emph{Universes} to properly define logical relations
\end{itemize}
\end{frame}

\begin{frame}{Outline}
\begin{enumerate}
\item Theory
\begin{itemize}
\item Design a DTT with built-in type of LF derivations
\item Design algorithm for \emph{definitional equality} incorporating equational theory of substitutions
\item Prove soundness, completeness, and termination of algorithm
\end{itemize}
\item Prototype implementation
\begin{itemize}
\item Adapt Beluga implementation to dependent types
\item Implement algorithm for definitional equality
\end{itemize}
\item Case studies
\begin{itemize}
\item Formalize advanced proofs using the prototype
\end{itemize}
% Say: Allows us to fill in checkmarks for last two problems?
\end{enumerate}
\end{frame}

\begin{frame}{Theory: Sketch}
\begin{itemize}
\item LFMTP'13 describes syntactic algorithm for equational theory of substitutions
\item Hurdle: definitional equality includes \textbf{computation} rules
\item DTT proof requires \textbf{semantic} techniques
\item Approach:
\begin{itemize}
\item Adapt proof for ``plain'' dependent type theory with a universe (e.g. [Abel '11])
\item Follow Allais et al. '13 to include ``non-computational'' laws
\item Byproducts of proof: normalization and consistency!
\end{itemize}
\end{itemize}
\end{frame}

\begin{frame}{Prototype implementation: Adapting Beluga}
Some progress already:
\begin{itemize}
\item Merging index and computation language contexts \good
\item Allowing indices to mention computations: in progress
\item Extending definitional equality: to do
\item Implementing universes: to do
\end{itemize}
\end{frame}

\begin{frame}{Case studies: Some advanced proofs to target}
\begin{enumerate}
\item Soundness and completeness of a normalization by evaluation algorithm [Dybjer and Filinski '00]
\item Normalization for Martin-L\"of type theory with a universe [Coquand '98]
\item Logical relation for contextual equivalence [Dreyer et al '11]
\item Normalization for System F [Girard '89]
\end{enumerate}

$\,$

\begin{itemize}
\item Challenging. 2\&{}3 have not been fully formalized (?)
\item Feasible. I have partially formalized 1\&{}2 in Agda
\item Would benefit a lot from proposed support for substitutions!
\item Significant fragments of real systems
\item 3 is cutting-edge: researchers \emph{want} to be able to formalize
\item Require expressive power beyond Beluga today
\end{itemize}
\end{frame}

% \begin{frame}{A solution?}
% \begin{itemize}
% \pause \item We know that LF works well to encode object languages...
% \pause \item So do this *once* for LF, and embed your OL inside
% \pause \item I've prototyped this in Agda
% \pause \item $>$48GB of memory to typecheck an example
% \pause \item And we still have to apply these lemmas manually and often compose several of them
% \pause \item Solution: build LF datatype into DTT and build lemmas into conversion check
% \pause \item Result: ``dependently typed Beluga''
% \pause \item Analyze LF objects by the induction principle you get from considering an inductive type representing LF
% \pause \item provides ``simple'' approach to semantics, termination (via eliminator for LF datatype)?
% \end{itemize}
% \end{frame}

% \begin{frame}{The metric for success}
% \begin{itemize}
% \item Algorithm for typechecking (decidability, soundness, completeness)
% \item Consequences: normalization, consistency
% \item Prototype implementation
% \item Formalize non-trivial proofs, e.g.
% \begin{itemize}
% \item C. Coquand's normalization proof for MLTT with a universe
% \item logical relation for contextual equivalence (parametricity)
% \item soundness and completeness of NbE
% \item normalization for System F?
% \end{itemize}
% \end{itemize}
% \end{frame}

\begin{frame}{Conclusion}
\begin{itemize}
\item Building a logical framework into dependent type theory appears feasible
\item Substantial benefits: formalizing advanced results in programming language theory becomes more practical
\item Hopefully not much more effort than writing a paper proof!
\item Hopefully provides insights into how to incorporate other equational theories from other areas of application?
\end{itemize}
\end{frame}

\end{document}

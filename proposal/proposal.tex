\documentclass{article}
\author{Andrew Cave}
\title{Dependent type theory for contextual reasoning}
\begin{document}
\maketitle

\section{Introduction}
Contrast dependent type theory with indexed types (Beluga) and first-order
logic (Abella). 

Thesis statement: Building a logical framework into dependent type
theory is possible and enables effective mechanization of state-of-the-art programming
language metatheory. (Moreover: it's a rich programming
environment. ``sliding scale'' of programming to verification with dependent types)
(todo: look at other examples of thesis statements)

clarify: this means e.g. logical relations, like Coquand's type
directed conversion (the ITP paper), soundness/completeness of NbE
(cite e.g. Abel for recent work, Dybjer for original?),
step-indexed logical relations for reasoning about effects

design space: systems with support for binding and substitution vs
systems with metatheoretic strength and rich abstraction mechanisms

main issues:
in first order representations, defining substitution principles

``A domain-specific dependent type theory''

though actually this isn't the worst of it, the worst of it is
establishing necessary equational properties of substitution and weakening
(freshness in the nominal world, shifts in the de Bruijn world) (cite
Altenkirch) and deducing equations from the resulting equational
theory.

classic: $[P/y]([N/x]M) = [([P/y]N)/x] ([P/y]M)$ provided $x\neq y$
(check this: Barendregt?)

higher order abstract syntax partially solves this

distinction between techniques for supporting ``names'', weakening, freshness (nominal) and ``substitution''.

dependent type theory allows effective programming and proving
\section{Background}
Sketch well-scoped representation in Agda and substitution
principles. Illustrate the point that the main bottleneck for
mechanization are the equations which arise (e.g. church rosser proof?
similar experiments as far as proving normalization for MLTT with a universe)

\subsection{Indexed recursive types}

don't quite suffice for proofs: need positivity. don't quite suffice
for logical relations: need universe (large eliminations) to justify,
since they aren't positive definitions
\subsection{First class substitutions}

\subsection{Example}
\section{Remaining to be done}
\subsection{Computation in types}
basic design sketch

why: richer tools (functions which compute instead of functional
relations) for more effective proofs

necessary to support state-of-the-art proofs involving polymorphism,
universes, ``generic programming'', dependent types, step indexing.

technique for induction principle: induct over all of ``LF''
parameterized by a signature, permitting mutual definitions, etc.
\subsection{Some theory}

issues: adequacy, decidability of typechecking, (strong) normalization
\section{Timeline}
Next 2-5 months: working out the metatheory and examples of computation in
types, potentially for a POPL paper in July.

backup/option: implementation (prototype?). ``structural''/``categorical'' account of a logical framework?

optional piece: support for (and examples of) step-indexed logical
relations

\begin{thebibliography}
Test
\end{thebibliography}
\end{document}